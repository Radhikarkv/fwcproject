\def\mytitle{FPGA  ASSIGNMENT}
\def\myauthor{R.Radhika}
\def\contact{r170234@rguktrkv.ac.in}
\def\mymodule{Future Wireless Communications (FWC)}
\documentclass[10pt, a4paper]{article}
\usepackage[a4paper,outer=1.5cm,inner=1.5cm,top=1.75cm,bottom=1.5cm]{geometry}
\twocolumn
\usepackage{graphicx}
\graphicspath{{./images/}}
\usepackage[colorlinks,linkcolor={black},citecolor={blue!80!black},urlcolor={blue!80!black}]{hyperref}
\usepackage[parfill]{parskip}
\usepackage{lmodern}
\usepackage{tikz}

\usepackage{karnaugh-map}

\usepackage{tabularx}
%\documentclass{article}
%\documentclass[tikz, border=2mm]{standalone}
%\usepackage{tikz}
\usepackage{circuitikz}
\usetikzlibrary{calc}

\renewcommand*\familydefault{\sfdefault}
\usepackage{watermark}
\usepackage{lipsum}
\usepackage{xcolor}
\usepackage{listings}
\usepackage{float}
\usepackage{titlesec}

\titlespacing{\subsection}{1pt}{\parskip}{3pt}
\titlespacing{\subsubsection}{0pt}{\parskip}{-\parskip}
\titlespacing{\paragraph}{0pt}{\parskip}{\parskip}
\newcommand{\figuremacro}[5]{
    \begin{figure}[#1]
        \centering
        \includegraphics[width=#5\columnwidth]{#2}
        \caption[#3]{\textbf{#3}#4}
        \label{fig:#2}
    \end{figure}
}

\lstset{
frame=single, 
breaklines=true,
columns=fullflexible
}

%\thiswatermark{\centering \put(400,-128.0){\includegraphics[scale=0.3]{logo}} }
\title{\mytitle}
\author{\myauthor\hspace{1em}\\\contact\\IITH\hspace{0.5em}-\hspace{0.6em}\mymodule}
\date{19-09-2022}
\begin{document}
  \maketitle
  \tableofcontents
  \begin{abstract}
      This manual shows that move the content of one register to another register  :
%\figuremacro{h}{diag}{}{}{0.9}
  \end{abstract}

  
\section{Introduction}
    \subsection{7474 IC:}
This IC contains 2 D-flip flops.\\
For this section total of 4 flip-flops(2 ICs) are required since we need to design a 4-bit shift register.

\subsection{Arduino:}
    In Arduino Uno we generate the clock pulse which is given to the each and every flip-flop by default.\\
    We take 5 volts and Ground as the supply to the bread board from the Arduino board.


  \section{Components}
  \begin{tabularx}{0.4\textwidth} { 
  | >{\centering\arraybackslash}X 
  | >{\centering\arraybackslash}X 
  | >{\centering\arraybackslash}X
  | >{\centering\arraybackslash}X | }
\hline
 \textbf{Component}& \textbf{Values} & \textbf{Quantity}\\
\hline
Arduino & UNO & 1 \\  
\hline
JumperWires& M-M & 20 \\ 
\hline
Breadboard &  & 1 \\
\hline
IC & 7447 &2 \\
\hline
\end{tabularx}

\section{PIN Diagram}
\begin{center}
   \includegraphics[width=3in]{pindiagram.png}
\end{center}


\begin{center}
Figure.a
\end{center}

\section{Truth Table}
  \begin{tabularx}{0.46\textwidth} { 
  | >{\centering\arraybackslash}X 
  | >{\centering\arraybackslash}X 
  | >{\centering\arraybackslash}X
  | >{\centering\arraybackslash}X 
  | >{\centering\arraybackslash}X 
  | >{\centering\arraybackslash}X 
  | >{\centering\arraybackslash}X 
  | >{\centering\arraybackslash}X 
  | >{\centering\arraybackslash}X 
  | >{\centering\arraybackslash}X | }


\hline
D1 & Q1=D2 & Q2=D3 & Q3=D4  & Q4\\
\hline
0 & 0 & 0 & 0 & 0 \\  
\hline
1 & 1 & 0 & 0 & 0 \\ 
\hline
1 & 1 & 1 & 0 & 0 \\
\hline
0 & 0 & 1 & 1 & 0 \\
\hline
0 & 0 & 0 & 1 & 1 \\  
\hline
0 & 0 & 0 & 0 & 1\\ 
\hline
0 & 0 & 0 & 0 & 0 \\
\hline
\end{tabularx}
\begin{center}
 Truth table for 0110
\end{center}

\section{Circuit Diagram}
\begin{figure}
\centering
    \includegraphics[width=4in]{SIPOregister.png}
\end{figure}
\begin{center}
Figure.b
 \end{center}
    \paragraph{4-bit shift register:}
 1.It has 4 D-flip flops.\\
 2.Verify the output for the sequence by changing the D1 pin to Vcc and Ground for different clock cycles.\\
 3.It has 4 outputs i.e Q1, Q2, Q3 and Q4.\\
 4. We need to give the input from MSB to LSB.\\
          \paragraph{Problem-1}
    1. Connect the circuit as per the above diagram.\\
    2. Execute the circuit using the below code.\\
	\paragraph{Problem-2}
1. Same circuit can be implemented by without IC display to the Q1, Q2, Q3 AND Q4 respectively.\\
2. Execute the circuit using the below code.\\

\begin{tabularx}{0.46\textwidth} { 
  | >{\centering\arraybackslash}X |}
  \hline
 https://github.com/Radhikarkv/fwcproject.git\\
  \hline
\end{tabularx}

\bibliographystyle{ieeetr}
\section{Circuit Implementation}
  \begin{tabularx}{0.86\textwidth} { 
  | >{\centering\arraybackslash}X 
  | >{\centering\arraybackslash}X 
  | >{\centering\arraybackslash}X
  | >{\centering\arraybackslash}X 
  | >{\centering\arraybackslash}X 
  | >{\centering\arraybackslash}X 
  | >{\centering\arraybackslash}X 
  | >{\centering\arraybackslash}X 
  | >{\centering\arraybackslash}X
  | >{\centering\arraybackslash}X
  | >{\centering\arraybackslash}X
  | >{\centering\arraybackslash}X
  | >{\centering\arraybackslash}X
  | >{\centering\arraybackslash}X
  | >{\centering\arraybackslash}X 
  | >{\centering\arraybackslash}X | }

\hline
\textbf{Ard} & \textbf{D13} & \textbf{D13} &  &  &  & \textbf{Vcc} & \textbf{Vcc} & \textbf{Vcc} & \textbf{Vcc} & \textbf{Vcc} & \textbf{Gnd} &  &  &  & \\  
\hline
\textbf{7474} & 3 & 11 & 5-12 & 9 &  & 1 & 4 & 10 & 13 & 14 & 7 & 5 & 9 &  &  \\
\hline
\textbf{7474} & 3 & 11 &  & 2 & 5-12 & 1 & 4 & 10 & 13 & 14 & 7 &  &  & 5 & 9  \\
\hline
\textbf{LED} &  &  &  &  &  &  &  &  &  &  &  & led1 & led2 & led3 & led4  \\
\hline
\end{tabularx}

\begin{center}
    Connections
\end{center}
   \section{Setup}
\begin{enumerate}
\item Connect the Vaman to the Laptop through USB.
\item There is a button and an LED to the left of the USB port on the Vaman.There is another button to the right of the LED.
\item Press the right button first and immediately press the left button.The LED will be blinking green.The Vaman is now in bootloader mode.
\end{enumerate}
\subsection{Steps for implementation}
\begin{enumerate}
\item Login to termux-ubuntu on the android device and execute the following commands:\\
Make sure that the required installation and tool builds of pygmy-sdk had done prior executing below commands
\begin{lstlisting}
proot-distro login debian
cd  /data/data/com.termux/files/home/
mkdir fpga
svn co https://github.com/Radhikarkv/fwcproject.git/trunk/fpga/codes
cd codes
ql_symbiflow -compile -src /data/data/com.termux/files/home/fpga/codes -d ql-eos-s3 -P PU64 -v helloworldfpga.v -t helloworldfpga -p quickfeather.pcf -dump binary
\end{lstlisting}
This will generate \textbf{helloworldfpga.bin} file in codes directory transfer this bin file to laptop by executing the following command
\begin{lstlisting}
scp /data/data/com.termux/files/home/fpga/codes/helloworldfpga.bin username_of_pc@IP_address:/home/username
\end{lstlisting}
Make sure that the appropriate username,IP address of the Laptop is given in the above command.
\item Now execute the following commands on the Laptop terminal\\
Make sure that required installation of programmer application had done prior executing below command
\begin{lstlisting}
python3 /home/username/TinyFPGA-Programmer-Application/tinyfpga-programmer-gui.py --port /dev/ttyACM0 --appfpga /home/username/helloworldfpga.bin --mode fpga
\end{lstlisting}
\end{enumerate}
\end{document}

