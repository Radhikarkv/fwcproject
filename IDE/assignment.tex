\def\mytitle{IDE  ASSIGNMENT}
\def\myauthor{R.Radhika}
\def\contact{r170234@rguktrkv.ac.in}
\def\mymodule{Future Wireless Communications (FWC)}
\documentclass[10pt, a4paper]{article}
\usepackage[a4paper,outer=1.5cm,inner=1.5cm,top=1.75cm,bottom=1.5cm]{geometry}
\twocolumn
\usepackage{graphicx}
\graphicspath{{./images/}}
\usepackage[colorlinks,linkcolor={black},citecolor={blue!80!black},urlcolor={blue!80!black}]{hyperref}
\usepackage[parfill]{parskip}
\usepackage{lmodern}
\usepackage{tikz}

\usepackage{karnaugh-map}

\usepackage{tabularx}
%\documentclass{article}
%\documentclass[tikz, border=2mm]{standalone}
%\usepackage{tikz}
\usepackage{circuitikz}
\usetikzlibrary{calc}

\renewcommand*\familydefault{\sfdefault}
\usepackage{watermark}
\usepackage{lipsum}
\usepackage{xcolor}
\usepackage{listings}
\usepackage{float}
\usepackage{titlesec}

\titlespacing{\subsection}{1pt}{\parskip}{3pt}
\titlespacing{\subsubsection}{0pt}{\parskip}{-\parskip}
\titlespacing{\paragraph}{0pt}{\parskip}{\parskip}
\newcommand{\figuremacro}[5]{
    \begin{figure}[#1]
        \centering
        \includegraphics[width=#5\columnwidth]{#2}
        \caption[#3]{\textbf{#3}#4}
        \label{fig:#2}
    \end{figure}
}

\lstset{
frame=single, 
breaklines=true,
columns=fullflexible
}

%\thiswatermark{\centering \put(400,-128.0){\includegraphics[scale=0.3]{logo}} }
\title{\mytitle}
\author{\myauthor\hspace{1em}\\\contact\\IITH\hspace{0.5em}-\hspace{0.6em}\mymodule}
\date{19-09-2022}
\begin{document}
  \maketitle
  \tableofcontents
  \begin{abstract}
      This manual shows that move the content of one register to another register  :
%\figuremacro{h}{diag}{}{}{0.9}
  \end{abstract}

  
\section{Introduction}
    \subsection{7474 IC:}
This IC contains 2 D-flip flops.\\
For this section total of 4 flip-flops(2 ICs) are required since we need to design a 4-bit shift register.

\subsection{Arduino:}
    In Arduino Uno we generate the clock pulse which is given to the each and every flip-flop by default.\\
    We take 5 volts and Ground as the supply to the bread board from the Arduino board.


  \section{Components}
  \begin{tabularx}{0.4\textwidth} { 
  | >{\centering\arraybackslash}X 
  | >{\centering\arraybackslash}X 
  | >{\centering\arraybackslash}X
  | >{\centering\arraybackslash}X | }
\hline
 \textbf{Component}& \textbf{Values} & \textbf{Quantity}\\
\hline
Arduino & UNO & 1 \\  
\hline
JumperWires& M-M & 20 \\ 
\hline
Breadboard &  & 1 \\
\hline
IC & 7447 &2 \\
\hline
\end{tabularx}

\section{PIN Diagram}
\begin{center}
   \includegraphics{pindiagram.png}
\end{center}


\begin{center}
Figure.a
\end{center}

\section{Truth Table}
  \begin{tabularx}{0.46\textwidth} { 
  | >{\centering\arraybackslash}X 
  | >{\centering\arraybackslash}X 
  | >{\centering\arraybackslash}X
  | >{\centering\arraybackslash}X 
  | >{\centering\arraybackslash}X 
  | >{\centering\arraybackslash}X 
  | >{\centering\arraybackslash}X 
  | >{\centering\arraybackslash}X 
  | >{\centering\arraybackslash}X 
  | >{\centering\arraybackslash}X | }


\hline
D1 & Q1=D2 & Q2=D3 & Q3=D4  & Q4\\
\hline
0 & 0 & 0 & 0 & 0 \\  
\hline
1 & 1 & 0 & 0 & 0 \\ 
\hline
1 & 1 & 1 & 0 & 0 \\
\hline
0 & 0 & 1 & 1 & 0 \\
\hline
0 & 0 & 0 & 1 & 1 \\  
\hline
0 & 0 & 0 & 0 & 1\\ 
\hline
0 & 0 & 0 & 0 & 0 \\
\hline
\end{tabularx}
\begin{center}
 Truth table for 0110
\end{center}

\section{Circuit Diagram}
\begin{figure}
\centering
    \includegraphics[width=4in]{SIPOregister.png}
\end{figure}
\begin{center}
Figure.b
 \end{center}
    \paragraph{4-bit shift register:}
 1.It has 4 D-flip flops.\\
 2.Verify the output for the sequence by changing the D1 pin to Vcc and Ground for different clock cycles.\\
 3.It has 4 outputs i.e Q1, Q2, Q3 and Q4.\\
 4. We need to give the input from MSB to LSB.\\
\section{Implementation}

\begin{center}
    Connections
\end{center}

        
    \paragraph{Problem-1}
    1. Connect the circuit as per the above diagram.\\
    2. Execute the circuit using the below code.\\
\begin{tabularx}{0.46\textwidth} { 
  | >{\centering\arraybackslash}X |}
  \hline
  https://github.com/Radhikarkv/fwcproject.git\\
  \hline
\end{tabularx}





	\paragraph{Problem-2}
1. Same circuit can be implemented by without IC display to the Q1, Q2, Q3 AND Q4 respectively.\\
2. Execute the circuit using the below code.\\

\begin{tabularx}{0.46\textwidth} { 
  | >{\centering\arraybackslash}X |}
  \hline
 https://github.com/Radhikarkv/fwcproject.git\\
  \hline
\end{tabularx}

\bibliographystyle{ieeetr}
\end{document}